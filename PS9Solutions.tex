\title{Math 55: Problem Set 9}
\documentclass[12pt,letterpaper]{article}

\usepackage{amsmath} 
\usepackage{amssymb}
\usepackage{ulem}
\usepackage{wasysym}
\usepackage{tikz}
\usepackage{multicol}
\usepackage{verbatim}
\usepackage[left=1in,top=1in,right=1in,bottom=1in,nohead]{geometry}
\usetikzlibrary{decorations.markings}
\usetikzlibrary{decorations.pathreplacing}

\usepackage{amsthm} 
\usepackage{wrapfig}
\usepackage{enumitem}
%\usepackage{enumerate}
\newtheorem{mydef}{Definition}
\newtheorem{example}{Example}
\newtheorem{thrm}{Theorem}
\newtheorem{lemma}{Lemma}
\newtheorem{cor}{Corollary}
\newtheorem{notation}{Notation}
\newtheorem{rem}{Remarks}
\newtheorem{obs}{Observation}
\newtheorem{claim}{Claim}
\newtheorem{term}{Terminology}
\newcommand{\biu}[1]{\underline{\textbf{\textit{#1}}}}
\newcommand{\so}{\Rightarrow}
\newcommand{\onlyif}{\Leftarrow}
\newcommand{\myiff}{\Leftrightarrow}
\usepackage[ampersand]{easylist}

\let\oldemptyset\emptyset
\let\emptyset\varnothing

\author{Keno Fischer}

\newcommand{\homework}{\biu{Homework}}
\newcommand{\Mor}{\text{Mor}}
\newcommand{\Hom}{\text{Hom}}
\newcommand{\clos}{\text{clos }}
%\newcommand{\Im}{\text{Im }}
\newcommand{\Ker}{\text{ker }}
\newcommand{\N}{\mathbb{N}}
\newcommand{\Q}{\mathbb{Q}}
\newcommand{\Z}{\mathbb{Z}}
\newcommand{\R}{\mathbb{R}}
\newcommand{\C}{\mathbb{C}}
\newcommand{\limn}{\lim\limits_{n\to\infty} }
\newcommand{\limk}{\lim\limits_{k\to\infty} }
\newcommand{\pabs}[1]{\left|\left| #1 \right|\right|_p}
\newcommand{\mesh}{\text{mesh}}
\newcommand{\set}[1]{\left{#1 \right}}
\newcommand{\paren}[1]{\left(#1 \right)}
\newcommand{\tensorp}{\bigotimes}
\newcommand{\tensor}{\otimes}
\newcommand{\Wedge}{\bigwedge}
\newcommand{\End}{\text{End}}
%\newcommand{\pabs}[1]{#1}
\begin{document}
\tikzstyle{lattice}=[shape=circle,draw,fill,text=white]
\tikzset{
  % style to apply some styles to each segment of a path
  on each segment/.style={
    decorate,
    decoration={
      show path construction,
      moveto code={},
      lineto code={
        \path [#1]
        (\tikzinputsegmentfirst) -- (\tikzinputsegmentlast);
      },
      curveto code={
        \path [#1] (\tikzinputsegmentfirst)
        .. controls
        (\tikzinputsegmentsupporta) and (\tikzinputsegmentsupportb)
        ..
        (\tikzinputsegmentlast);
      },
      closepath code={
        \path [#1]
        (\tikzinputsegmentfirst) -- (\tikzinputsegmentlast);
      },
    },
  },
  % style to add an arrow in the middle of a path
  end arrow/.style={postaction={decorate,decoration={
        markings,
        mark=at position 1 with {\arrow[#1]{stealth}}
      }}},
}

\maketitle
\section*{Problem 1}
\subsection*{a)}
Let $k$ be a field, $K$ a finite extension of $k$ and let $L$ be a finite extension of $K$ (i.e. $k \subset K \subset L$).  \par
Now, since $K$ is a finite extension of $k$, there exist a finite basis of elements $v_K^{(1)},\ldots,v_K^{(n)}$, $v_K^{(i)}\in K$, where $n=[K:k]$ is the dimension of $K$, considered as a vector space over $K$. Similarly, there exists a finite basis $v_L^{(1)},\ldots,v_L^{(n)}$, $v_L^{(i)}\in L$ of $L$ where $m=[L:K]$. \par 
Now, consider the set $B=\{ v_K^{(i)}v_L^{(1)} | 1\leq i\leq n, 1\leq j\leq m\}$ where we consider $v_K^{(i)},v_L^{(1)}$ as elements in $L$ (which we may do since $K\subset L$). Now, recall that we may write any element in $L$ as $v=\sum_{i=1}^n a_K^{i} v_L^{(i)}$ for some $a_K^{1},\ldots,a_K^{n}\in K$. However, since $a_K^{n}\in K\subset L$, we may write each $a_K^{n}$ as $\sum_{j=1}^m a_j v_K^{(j)}$ and thus we we may express any element $v\in L$ as $v=\sum_{i=1}^n \sum_{j=1}^m a_{ij} v_K^{(j)} v_L^{(i)}$ and thus $B$ spans $L$. To show that $B$ is linearly independent consider, $\sum_{i=1}^n \left(\sum_{j=1}^m a_{ij} v_K^{(j)}\right) v_L^{(i)}$. Since the $v_L^{(i)}$ are linearly independent, we must have $\sum_{j=1}^m a_{ij} v_K^{(j)}=0$ for each $i$. However, since the $v_K^{(j)}$ are linearly independent, we must have $a_ij = 0$ for all $i,j$. Now, $\sum_{i=1}^n \left(\sum_{j=1}^m a_{ij} v_K^{(j)}\right) v_L^{(i)}=\sum_{i=1}^n \sum_{j=1}^m a_{ij} \left(v_K^{(j)} v_L^{(i)}\right)$ by associativity and thus $B$ is lineraly independent. Now, since $B$ (which has $mn$ elements) is both spanning and linearly independent it is a basis of $L$ as seen as a vector space over $k$ and thus we must have $[L:k]=[L:K][K:k]$.
\subsection*{b)}
Let $K$ be a field and let $L$ be a finite extension of $K$ such that $n=[L:K]$
For $a\in L$, consider the set $A=\{ a^i | 1\leq i \leq k \}$ (note that all elements of $A$ are in $L$ for any finite $k>n$. Then, since the cardinality of $A$ is greater than $n$, there must exist a linear relationship between some of the elements $a^i\in A$. I.e. there exist, $b_i\in K$, $1\leq i \leq N$ for some $N$ (since L is a vector space over $K$), not all equal to zero, such that $\sum_{i=1}^k b_ia_i = \sum_{i=1}^n b_ia^i=0$. Now let $j$ be the largest index such that $b_j \neq 0$. Then, letting $c_i=\frac{b_i}{b_j}$ (which we may do since $K$ is a field), we have found a monic polynomial given by $p(X)=\sum_{i=1}^k c_iX^i$ such that $p(A)=0$ as desired.
\subsection*{c)}
Let $K$ be a field, $L$, be a (not necessarily finite) field extension of $K$ and let $a\in L$  be Algebraic over $K$. We will first show that there exists an irreducible polynomial over $K$ such that $p(a)=0$. First note that since $a$ is algebraic, there must exist a monic polynomial $p_1(K)$ of finite degree $N$, not necessarily irreducible such that $p_1(a)=0$. Now, let $A$ be the set of all integers less than or equal to $N$ such that there exists such that for each $n\in A$, there exists at least one polynomial of degree $n$. Note that $A$ must be non-empty since it at least contains $N$ and that it must be finite, since it's a subset of all integers less than $N$ which is a finite set. Now, since $A$ is a finite set, there must a minimal element of $A$. Let $n$ be that minimal element and let $p(X)$ be a polynomial of that degree. Clearly $p(X)$ must be irreducible since otherwise we could find polynomials $p_2(X),p_3(X)$ such that $p(X)=p_2(X)p_3(X)$ where both $p_2$ and $p_3$ are non-constant and thus of degree greater than $0$, but less than that of $p(X)$. \par 
We will now proof that such a $p(X)$ is unique. For that purpose assume that there exists another irreducible polynomial $q(X)$ such that $q(A)=0$. Now, by the Euclidean algorithm, we may find polynomials $s(X),r(X)$ such that $q(X)=s(X)p(X)+r(X)$. Now, since  $p(a)=0$, we must have $r(a)=0$ ans since $q$ is irreducible, we must have either have $r(X)$ not be the zero polynomial or $s(X)=1$. Now, clearly if $r(X)$ is not the zero polynomial we have found a polynomial that has degree less than $n$, but still evaluates to $0$ at $0$, contradicting the minimality of $p(X)$ and thus we must have $s(X)=1$ and $q(X)=1\cdot p(X) = p(X)$. and thus $p(X)$ is unique. 
\subsection*{d)}
Let $K$ be a field, $L$ a (not necessarily finite) field extension of $K$. Then, for $a\in L$, let $p(X)$ be the unique irreducible polynomial of degree $n$ such that $p(a)=0$ and let $K(a)$ be the set of all values generated by polynomials $q(X)\in K[X]$ evaluated at $a$. We will prove that $K(a)$ is a finite extension of $K$. \par 
Note that $K[X]$ is commutative by the commutativity of $K$ and note that it is a PID and thus we may consider the ideal generated by p(X) as:
\[ I=\{ q(X)p(X) | q(X)\in K[X] \} \]
Now, consider $K[X]/I$ as vector space over $K$ in a natural way (where scalar multiplication is equivalent to multiplication by the constant polynomial). \par 
Let us now distinguish two cases. First if $a\in K$, $p(X)=a$, so $K(a)=K$, so we may consider $K(a)$ as a degree 1 field extension of $K$ and we are done. \par
Otherwise, note that the space $K[X]/I$, is spanned by the polynomials $\{X^i \in K[X] | i<n\}$ (by the fact that $K[X]$ is a PID, which easy to see if we consider the equivalence classes of polynomials with equal remainder by the euclidean algorithm since the degree of the remainder will always be less than that of $p(X)$). Now, consider the evaluation of the space $K[X]/I$ at $a$ and note that this space is just $K(a)$ (since $p(a)=0$) and that the the spanning set is just $\{a^i | i<n\}$.  Now, for all $i<n$, there may not exist a non-trivial linear relation $a^i\notin K$ by elements in $K$, for otherwise there would exist an irreducible polynomial of degree $i$, which cannot be the case by the previous problem and thus the set $\{ 1, a, a^2, \ldots, a^{n-1} \}$ is linearly independent over $K(a)$ (since  (since if $ba=c\in K$, then $a=b^{-1}c\in K \Rightarrow\Leftarrow$) and thus we have found a basis for $K(a)$ which contains $n$ elements and thus $K(a)$ is a finite extension of $K$ with degree $n$ which is also the degree of $p(X)$ as desired. 
\subsection*{e)}
We will first show that $p(X)=X^3-2$ is irreducible. To do so note that if there were to be polynomials whose product is equal to $p(X)$, they would have to be of the form $cX^2+aX^+b$ and $c^{-1}X+d$. multiplying this out and requiring that the coefficients in front of $X^2$ and $X$ be $0$, we find that we must have $\frac{a^3}{c^3}=e^3=\frac{f^3}{g^3}=2$ for some integers $f,g$. Clearly we must have $g=1$, the equation is reducible if there exists an integer $f$ such that $f^3=2$. However, this can clearly not be the case as $|f|<|f^3|$, thus we would have to have $|f|<2$, but the only such integer is $1$ for which $1^3=1$, so there exists no such integer and thus $X^3-2$ is irreducible. \par
From the previous problem set we know that any constructible number will lie in a field K that has $[K:\Q]=2n$. From part d) however, we know that the cube root of 2 lies in a field extension that has $[F:\Q]=3$ and therefore by part a). Furthermore, by part  a), there can exist no other constructible field $F_1$ such that $[F:F_1]\in {1,2}$ and therefore the cube root of three is not constructible. 
\subsubsection*{f)}
Since $\pi$ is transcendental over $\Q$, there may not exist extension (either finite or infinite) that contains $\pi$ (since otherwise $\pi$ would be algebraic by part b). However, since any constructible number must be the member of some extension as shown on the previous problem set, $\pi$ is not constructible. 
\section*{Problem 2}
\subsection*{a)}
Let $X$ be a compact Hausdorff space and let $C(X)$ be the set of all continuous functions with values in $\R$. Now consider the ideal $I_x\subset C(X)$ given by
\[ I_x = \{ f\in C(X) | f(x) = 0 \} \]
Now, let $f\in C(X)$ be some function such that $f(x)=b\neq 0$ and let $I'$ be the ideal given by adding $f$ to $I_x$. Now, since subtraction is continuous in $\R$, we must have $(f-b)\in C(X)$ and $(f-b)\in I_x$ since $(f-b)(x) = f(x)-b = 0$. Now, by closure under subtraction and multiplication (and since the constant function $\frac{1}{b}$ is in $C(X)$, since all constant functions are by an earlier problem set), we must have the multiplicative identity $\left(f(x)-(f-b)(x)\right)\frac{1}{b}(x)=1_\R \in I'$ and thus we must have $I' = C(X)$ and thus $I_x$ is a maximal ideal.
\subsection*{b)}
Let $I$ be a proper ideal of $C(X)$. Let us first proof that if $\{ x | f(x) = 0 \forall f \in I \}=\emptyset$, there is a function that is non-zero at all $x\in X$. Let $f,g$ be any two function in $I$ and consider $f^{-1}(\{0\})$, $g^{-1}(\{0\})$, both of which must be closed. Then if $(f^2+g^2)^{-1}(\{0\})=f^{-1}(\{0\})\cap g^{-1}(\{0\})=\emptyset$ for some $f,g\in C(X)$, we have found the desired function. Otherwise we know that the null set satisfies the FIP and thus, since $X$ is compact the total intersection of all $f^{-1}(X)$ must be non-empty and thus all the functions in $I$ must have some $0$ in common. \par 
Now, assume that $\{ x | f(x) = 0 \forall f \in I \}=\emptyset$. Then there exists a functions $f\in I$ such that $f(x)\neq 0$ for all $x\in A$. Then by continuity of reciprocation in $\R*$ ($f(x)$ has values in $\R*$ since it is never $0$), we must have $\frac{1}{f}(x) \in C(X)$ and so we must have $f(x)\frac{1}{f}(x) = 1 \in I$ and thus we must have $I=C(X)$ contrary to $I$ being a proper ideal. Thus we must have $A=\{ x | f(x) = 0 \forall f \in I \}\neq\emptyset$. Furthermore note that $A$ has one element if and only if $I$ is maximal (for if $A=\{x,y\}$ adding any element of $I_y$ that is not in $I_x$ to $I$ will make $I$, $I_y$ which is maximal and $A=\{y\}$).
Now, by part a) we have an injective map from $X$ to the set of all maximal ideas. By the above, every maximal ideal has a unique point in $X$ at which all functions vanish and thus there exists a an injective map between the maximal ideals and $X$.
\subsection*{c)}
We have already shown that it is possible to recover $X$ as a set via a bijection from $C(X)$. What remains to be shown is that we can also recover the notion of open sets. To do so consider the minimal topology on $X$ (seen as the set recovered by the maximal ideals above) such that all function in $C(X)$. We may construct this topology explicitly by considering $\tau^{C(X)}_{X} = \{  f^{-1}(A) | f\in C(X), A\in \tau_\R\}$. Since all functions in $C(X)$ are continuous on the original topology, we must have $\tau^{C(X)}_{X} \subseteq \tau^{X}_X$. Now, recall that $\{0\} \in \R$ is a closed set, so $R-\{0\}$ is an open set. Now, for any open set $B$ in the original topology on $X$, we may find it's adjoint closed set X-B and by Urysohn's lemma, there exists a function $F(x)$ s.t. $F(X-B) = \{0\}$ and thus w	e must have $X-F^{-1}(\{0\}) = B$ be an open set in the new topology generated by $C(X)$. 

\section*{Problem 3}
\subsection*{a)}
Let $p$ a projection operator. Now, consider $q=1_v-p$. We have  $q^2(v)=(1_v-p)(1_v-p)(v) = (1_v-p)(v-pv) = v-pv-(pv-p^2v)=v-pv=q(v)$ and thus $q$ is a projection operator.
\subsection*{b)}
Let $v\in V$. Then we may write $v$ as $v=pv+(v-pv)$. Now, we know that $pv\in \Im V$ by definition. However, also note that $p(v-Pv) = p(v)-p^2(v) = 0$, so $v-Pv\in \Ker p$. Therefore $\Im p$ and $\Ker p$ span $V$. What remains to be shown is that $\Im p \cap \Ker p = \{0\}$ to make this a direct product space. To show this, consider $x\in \Im p \cap \Ker p$ (which is clearly non-empty as it at least contains $0$). Now, there must exist $y$ such that $Py=x$, but we must also have $Px=0$. Now $Py=P^2 y=P(Py) = Px = 0$ so $x=0$ and $\Im p \cap \Ker p = \{0\}$ and $V=\Im p \bigoplus \Ker p$.
\subsection*{c)}
Let $W_1, W_2$ be subspaces of $V$ such that $V=W_1\bigoplus  W_2$. Then any $v$ is given uniquely by $v=w_1+w_2$ for some $w_1,w_2 \in W_1,W_2$. Now, let $p(v) = w_1$. Then $W_1=\Im p$, $W_2=\Ker P$ as desired. 
\subsection*{d)}
Let $V$ be a vector spaces, $W_1, W_2$ subspaces such that $V=W_1\bigoplus  W_2$ and let $T: V\to V$ be a linear transformation. 
\subsubsection*{Only If}
Let $W_1$ and $W_2$ be $T$-stable. We have $(T\circ p)(W_2) = T_2(\{0\})=\{0\}$ since $T$ is a linear transformation. We also have $(p\circ T)=p(W_2)=\{0\}$. $(T\circ p)(w_1)=T(w_1)=w_2$, $(p\circ T)(w_1)=p(w_2) = w_2$ for arbitrary $w_1,w_2\in W_1$ such that $w_2=Tw_1$ (which must exist since $W_1$ is T-stable). Now, since $T$ is linear, this is sufficient to show that $T$ and $p$ commute for any $v\in V$. 
\subsubsection*{If}
Conversely, If $T$ and $p$ commute, for $v=w_1+w_2\in V$, Now, $w_1 = p(v)$,so $T(p(v)) = p(T(V))\in W_1$, so $W_1$ is $T$-stable. Furthermore, we may not have $T(w_2)\in W_1-\{0\}$, for otherwise we would have $T(p(w_2))=p(T({0}))\in W_1-\{0\}$, contradicting the linearity of of $T$ and thus $W_2$ must be $T$-stable.
\end{document}