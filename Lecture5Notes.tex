\documentclass[12pt,letterpaper]{article}

\usepackage{amsmath} 
\usepackage{amssymb}
\usepackage{ulem}
\usepackage{wasysym}
\usepackage{tikz}
\usepackage{multicol}
\usepackage{verbatim}
\usepackage[left=1in,top=1in,right=1in,bottom=1in,nohead]{geometry}
\usetikzlibrary{decorations.markings}
\usetikzlibrary{decorations.pathreplacing}

\usepackage{amsthm} 
\usepackage{wrapfig}
\usepackage{enumitem}
%\usepackage{enumerate}
\newtheorem{mydef}{Definition}
\newtheorem{example}{Example}
\newtheorem{thrm}{Theorem}
\newtheorem{lemma}{Lemma}
\newtheorem{cor}{Corollary}
\newtheorem{notation}{Notation}
\newtheorem{rem}{Remarks}
\newtheorem{obs}{Observation}
\newtheorem{claim}{Claim}
\newtheorem{term}{Terminology}
\newcommand{\biu}[1]{\underline{\textbf{\textit{#1}}}}
\newcommand{\so}{\Rightarrow}
\newcommand{\onlyif}{\Leftarrow}
\newcommand{\myiff}{\Leftrightarrow}
\usepackage[ampersand]{easylist}

\let\oldemptyset\emptyset
\let\emptyset\varnothing

\author{Keno Fischer}

\newcommand{\homework}{\biu{Homework}}
\newcommand{\Mor}{\text{Mor}}
\newcommand{\Hom}{\text{Hom}}
\newcommand{\clos}{\text{clos }}
%\newcommand{\Im}{\text{Im }}
\newcommand{\Ker}{\text{ker }}
\newcommand{\N}{\mathbb{N}}
\newcommand{\Q}{\mathbb{Q}}
\newcommand{\Z}{\mathbb{Z}}
\newcommand{\R}{\mathbb{R}}
\newcommand{\C}{\mathbb{C}}
\newcommand{\limn}{\lim\limits_{n\to\infty} }
\newcommand{\limk}{\lim\limits_{k\to\infty} }
\newcommand{\pabs}[1]{\left|\left| #1 \right|\right|_p}
\newcommand{\mesh}{\text{mesh}}
\newcommand{\set}[1]{\left{#1 \right}}
\newcommand{\paren}[1]{\left(#1 \right)}
\newcommand{\tensorp}{\bigotimes}
\newcommand{\tensor}{\otimes}
\newcommand{\Wedge}{\bigwedge}
\newcommand{\End}{\text{End}}
%\newcommand{\pabs}[1]{#1}
\begin{document}
\tikzstyle{lattice}=[shape=circle,draw,fill,text=white]
\tikzset{
  % style to apply some styles to each segment of a path
  on each segment/.style={
    decorate,
    decoration={
      show path construction,
      moveto code={},
      lineto code={
        \path [#1]
        (\tikzinputsegmentfirst) -- (\tikzinputsegmentlast);
      },
      curveto code={
        \path [#1] (\tikzinputsegmentfirst)
        .. controls
        (\tikzinputsegmentsupporta) and (\tikzinputsegmentsupportb)
        ..
        (\tikzinputsegmentlast);
      },
      closepath code={
        \path [#1]
        (\tikzinputsegmentfirst) -- (\tikzinputsegmentlast);
      },
    },
  },
  % style to add an arrow in the middle of a path
  end arrow/.style={postaction={decorate,decoration={
        markings,
        mark=at position 1 with {\arrow[#1]{stealth}}
      }}},
}

\begin{term}
Suppose $(X,d)$ is a metric space $S\subset X$ a subset thereof. 
\begin{enumerate}
\item An open cover of $S$ is a collection of open subsets of $X$, $\{ U_\alpha|\alpha \in A\}$ such that $\cup_{a\in A} U_\alpha \supset S$.
\item A collection of subsets $\{C_\alpha | \alpha \in A \}$ of $x$ has the finite intersection property "FIP" if any finite subcollection of  has a non-empty intersection.
\item Let $(x_n)_{n\geq 1}$ be a sequence in $X$. Then a subsequence is a sequence of the form $(x_{n_k})$ with $n_1<n_2<\ldots$ 
\end{enumerate}
\end{term}
\begin{mydef}
$S$ is compact if every open cover of $S$ has a finite subcover. \par
$S$ is sequentially compact if every sequence $(x_n)$ in $S$ has a subsequence converging to a point in $S$. 
\end{mydef}
\begin{thrm}
For a subset $S\subset X$, the following are equivalent.
\begin{enumerate}
\item S is compact
\item Given any collection of closed subsets of $X$ $\{ C_\alpha | \alpha \in A \}$ such that $\{C_\alpha \cap S | \alpha \in A\}$ has the FIP, then $\cap_{\alpha \in A} (C_\alpha \in S)$ is non-empty.
\item S is sequentially compact
\end{enumerate}
\end{thrm}
\begin{rem}
\item In effect we we have two definitions of compactness
a) as stated ("extrinsic") \par
b) $S$ as a subset of $(S,d|_s)$ ("intrinsically defined")\par 
They are the same as follows from the following observation:\par 
A subset $T\subset S$ is open with resepect to $(S,d|_s)$ if and only if there exists a an open subset $U\subset X$ s.t. $U\cap S=T$.
\begin{proof}
\biu{$\so$} Suppose $T\subset S$ is open with resepect to $(S,d|_s)$. Then for each $t\in T$, there exists $\epsilon_t >0$ such that $B_S(t,\epsilon_t)\subset T$. Then $U=\cup_{t\in T} B_X(t,\epsilon_t)$ is open in $X$ and $U\cap S=T$.
\biu{$\onlyif$} Suppose $U\subset X$ is open and $U\cap S=T$. Then for each $t\in T$, there exists $\epsilon_t>0$ s.t. $B(t,\epsilon_t)\subset U$. Then $U \cap S = \cup_{t\in T} B_s(t,\epsilon_t) = T$.
\end{proof}
\item The definition of compactness, sequential compactness make sense in any topological space, but the implication that sequential compactness $\so$ compactness does not hold in general:
\end{rem}
\begin{proof}
$1) \Leftrightarrow 2)$ holds true for formal reasons:
There exists a bijectiion between open subsets $U_\alpha \subset X$ and closed subsets $C_\alpha\subset X$, Given by $U_\alpha = X-C_\alpha$. Then,\par
$\{U_\alpha|\alpha\in A\}$ is an open cover $\Leftrightarrow \cap_{\alpha\in A} (c_\alpha \cap S) = \emptyset$ \par 
$\{ U_\alpha | \alpha \in A\}$ has no finite subcover $\Leftrightarrow \{C_\alpha \cap S|\alpha\in A\}$ has the FIP. \par 
So 2) is the contrapositive of 1)\par
1) $\so$ 3). Let $(x_n)_{n\geq 1}$ be a sequence in $S$. Suppose there exists $x_\infty \in S$ such that for every $\epsilon>0$, $B(x_\infty,\epsilon)$ contains $x_n$ for infinitely many $n$. Then we can choose $(n)$ s.t.
\[ x_{n_1} \in B(x_\infty,1) \]
\[ x_{n_2} \in B\left(x_\infty,\frac{1}{2}\right) \]
\[ x_{n_k} \in B\left(x_\infty,\frac{1}{k}\right)\]
Then $\limk x_{n_k} = x_\infty$. \par 
If, On the other hand for each $x\in S, \exists \epsilon_x> 0$ such that $B(x,\epsilon_x)$ contains $x_n$ for  only finitely many $n$. Then, by compactness finitely many of the $B(x,\epsilon_x)$ cover $S$ and since $S$ contains $x_n$ for all $n$, at least one of the finitely many balls must contain infinitely many $x_n$ (infinitely many elements in finitely many balls). \biu{Contradiction}.
Therefore one must always be be able to find a subsequence by the former method. \par 
3) $\so$ 1) Will be a consequence of 3 lemmas
\end{proof}
\begin{mydef}
A subset $S \subset X$ is totally bounded if, for each $\epsilon>0$, there exists a finite subset $\{x_1,\ldots,x_n\}\subset S$. s.t. $\cup_{1\leq k \leq N} B(x_k,\epsilon) \supset S$. 
\end{mydef}
\begin{mydef}
S is countably compact if any countable open cover has a finite subcover.
\end{mydef}
\begin{lemma}
If $S$ is sequentially compact, then $S$ is totally bounded. 
\begin{proof}
Suppose $S$ is sequentially compact and $\epsilon>0$ given.
\begin{claim}
There cannot exist a sequence $(x_n)_{n\geq 1}$ in $S$ s.t. $B(x_n,\frac{\epsilon}{2}) \cap B(x_m,\frac{\epsilon}{2}) \cap S \neq \emptyset$ for all $n\neq m$
\begin{proof}
Suppose otherwise. Going to a convergent (in $S$) subsequence, without changing notation, we may suppose $x_n\to x\infty \in S$. Then $B(x_\infty,\frac{\epsilon}{2})$ must contain all but finitely many $x_n$ and therefore $x_\infty \in B(x_n,\frac{\epsilon}{2})$ for infinitely many $n$. Therefore $B(x_m,\frac{\epsilon}{2})\cap B(x_m,\frac{\epsilon}{2})$ contains $x_\infty$ for some $m\neq n \so$ Claim.
\end{proof}
\end{claim}
\end{proof}
\end{lemma}

\end{document}