\documentclass[12pt,letterpaper]{article}

\usepackage{amsmath} 
\usepackage{amssymb}
\usepackage{ulem}
\usepackage{tikz}
\usepackage[left=1in,top=1in,right=1in,bottom=1in,nohead]{geometry}
\usetikzlibrary{decorations.markings}
\usetikzlibrary{decorations.pathreplacing}

\usepackage{amsthm} 
\usepackage{wrapfig}
\usepackage{enumitem}
%\usepackage{enumerate}
\newtheorem{mydef}{Definition}
\newtheorem{example}{Example}
\newtheorem{thrm}{Theorem}
\newtheorem{lemma}{Lemma}
\newtheorem{cor}{Corollary}
\newtheorem{notation}{Notation}
\newtheorem{rem}{Remarks}
\newcommand{\biu}[1]{\underline{\textbf{\textit{#1}}}}
\newcommand{\so}{\Rightarrow}
\usepackage[ampersand]{easylist}

\let\oldemptyset\emptyset
\let\emptyset\varnothing

\newcommand{\homework}{\biu{Homework}}
\newcommand{\Mor}{\text{Mor}}
\newcommand{\N}{\mathbb{N}}
\newcommand{\Q}{\mathbb{Q}}
\newcommand{\Z}{\mathbb{Z}}
\newcommand{\R}{\mathbb{R}}
\newcommand{\C}{\mathbb{C}}
\newcommand{\pabs}[1]{\left|\left| #1 \right|\right|_p}
%\newcommand{\pabs}[1]{#1}
\begin{document}
\tikzstyle{lattice}=[shape=circle,draw,fill,text=white]
\tikzset{
  % style to apply some styles to each segment of a path
  on each segment/.style={
    decorate,
    decoration={
      show path construction,
      moveto code={},
      lineto code={
        \path [#1]
        (\tikzinputsegmentfirst) -- (\tikzinputsegmentlast);
      },
      curveto code={
        \path [#1] (\tikzinputsegmentfirst)
        .. controls
        (\tikzinputsegmentsupporta) and (\tikzinputsegmentsupportb)
        ..
        (\tikzinputsegmentlast);
      },
      closepath code={
        \path [#1]
        (\tikzinputsegmentfirst) -- (\tikzinputsegmentlast);
      },
    },
  },
  % style to add an arrow in the middle of a path
  end arrow/.style={postaction={decorate,decoration={
        markings,
        mark=at position 1 with {\arrow[#1]{stealth}}
      }}},
}


\begin{lemma}
\biu{Zorn's Lemma} Suppose a partially ordered set $P$ has the property that every totally ordered subset has an upper bound in $P$. Then the set $P$ contains at least one maximal element.
\section{Problems}
\begin{itemize}
\item Prove that if $A$ is an infinite set, then $A$ and $A\times \N$ have the same cardinality.
\begin{proof}
We know from previous exercise that there exists a bijective map $\Phi: \N\to\N\times\N$. Define
\[ F = \{ f_0 | f_0:X\to X\times\N, X\subseteq A, f_0 \text{is a bijection}\}\]
Now note that since $A$ is infinite, there must exist $S\subset A$ s.t. $\exists\; g:S\xrightarrow{\text{bij.}}\N$. 
Furthermore note that that if there exists a bijective map  $\Omega: A\to B$, then there exists a bijective map $\Omega^*:A\times C\to B\times C$ given by 
$\Omega^*(x,c)=(\Omega(x),c)$. Given this information we can see that the map $g^{-1}\Phi g$ must be a bijective map from $S$ to $S\times \N$ and thus  $g^{-1}\Phi g\in F$ so $F$ is nonempty.

We can now go on to define a partial ordering on $F$ as follows: If $f$ is an extension of $g$, then $f>g$. Note that that this ordering on $F$ satisfies Zorn's lemma as the domain of any map in $F$ is bounded by $A$.

We may therefore choose a maximal element in $F$. Let  $f*: B\to B\times \N$ be that element. Now there are two cases
\begin{enumerate}
\item A-B is finite\par
Note that $B$ is infinite and thus there must be a copy of $\N$ in $B$ and we may write $B$ as $B_1\cup \N$.
We now find find a bijective map $f: B\to A$ as follows. For values in $B_1$ we have have $f(b)=b$ and for values in $\N$ we have $f: \N\to (A-B)\cup\N$, which can be easily found since $A-B$ is finite and can thus be just prepended to $\N$. Note that $f$ is bijective.

We may now define a bijective map $h: A\to A\times \N$ as $f^{-1}f^*f$
 ($A\xrightarrow{f^{-1}}B\xrightarrow{f^*}B\times\N\xrightarrow{f}A\times\N$)

\end{enumerate}

\end{proof}
\end{itemize}
\end{lemma}

\end{document}