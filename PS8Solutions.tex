\documentclass[12pt,letterpaper]{article}

\usepackage{amsmath} 
\usepackage{amssymb}
\usepackage{ulem}
\usepackage{tikz}
\usepackage[left=1in,top=1in,right=1in,bottom=1in,nohead]{geometry}
\usetikzlibrary{decorations.markings}
\usetikzlibrary{decorations.pathreplacing}

\usepackage{amsthm} 
\usepackage{wrapfig}
\usepackage{enumitem}
%\usepackage{enumerate}
\newtheorem{mydef}{Definition}
\newtheorem{example}{Example}
\newtheorem{thrm}{Theorem}
\newtheorem{lemma}{Lemma}
\newtheorem{cor}{Corollary}
\newtheorem{notation}{Notation}
\newtheorem{rem}{Remarks}
\newcommand{\biu}[1]{\underline{\textbf{\textit{#1}}}}
\newcommand{\so}{\Rightarrow}
\usepackage[ampersand]{easylist}

\let\oldemptyset\emptyset
\let\emptyset\varnothing

\newcommand{\homework}{\biu{Homework}}
\newcommand{\Mor}{\text{Mor}}
\newcommand{\N}{\mathbb{N}}
\newcommand{\Q}{\mathbb{Q}}
\newcommand{\Z}{\mathbb{Z}}
\newcommand{\R}{\mathbb{R}}
\newcommand{\C}{\mathbb{C}}
\newcommand{\pabs}[1]{\left|\left| #1 \right|\right|_p}
%\newcommand{\pabs}[1]{#1}
\begin{document}
\tikzstyle{lattice}=[shape=circle,draw,fill,text=white]
\tikzset{
  % style to apply some styles to each segment of a path
  on each segment/.style={
    decorate,
    decoration={
      show path construction,
      moveto code={},
      lineto code={
        \path [#1]
        (\tikzinputsegmentfirst) -- (\tikzinputsegmentlast);
      },
      curveto code={
        \path [#1] (\tikzinputsegmentfirst)
        .. controls
        (\tikzinputsegmentsupporta) and (\tikzinputsegmentsupportb)
        ..
        (\tikzinputsegmentlast);
      },
      closepath code={
        \path [#1]
        (\tikzinputsegmentfirst) -- (\tikzinputsegmentlast);
      },
    },
  },
  % style to add an arrow in the middle of a path
  end arrow/.style={postaction={decorate,decoration={
        markings,
        mark=at position 1 with {\arrow[#1]{stealth}}
      }}},
}

\section*{Problem 1a)}
\subsection*{Existence}
Let $D$ be an integral domain and consider the collection 
\[ A = \{ d | d\in D-\{0\} \text{ s.t. } d^{-1} \notin D \} \cup \{e\} \]
. Note that there exists an invective map $p: A\to D$ given by $p(a)=d$
We will define multiplication on $A$ to be the same as multiplication on $D$ (and it thus inherits commutativity from mutliplication on $A$). Furthermore note that multiplication on $d$ is is closed by the following observation: \par $d_1,d_2 \in D$ Assume $(d_1d_2)^{-1}4$, but $d_1^{-1}, d_2^{-1} \notin D$. Then we have $e=(d_1d_2)^{-1}(d_1d_2)= d_1((d_1d_2)^{-1}d_2)$, but that would imply that $d_1^{-1} = (d_1d_2)^{-1}d_2 \in D$ by closure of multiplication in $D$ which contradicts the assumption that $d_1^{-1}\notin D$. \par 
Now consider 
\[ K = (D \times A)/\sim\]
Where $\sim$ is an equivalence relation given by $(d_1,a_1)\sim (d_2,a_2) \iff d_1p(a_2) = d_2p(a_1)$. For convenience, we will write $\frac{d}{a}$ for an element $(d,a)\in K$. We will define the binary multiplication operation $\cdot$ as follows:
 \[\frac{d_1}{a_1} \cdot \frac{d_2}{a_2} = \frac{d_1\cdot d_2}{a_1\cdot da2} = \frac{d_1d_2}{a_1a_2}.\]
 Note that this operation must be closed, since multiplication is closed in both $A$ and $D$, it must be commutative by the commutativity of multiplication in both $A$ and $D$, and it has a unit given by $e_K=\frac{e_A}{e_D}e]$. We will furthermore define inversion as follows:
\[ \left(\frac{d}{a}\right)^{-1} = \begin{cases} \dfrac{p(a)}{p^{-1}(d)} &\mbox{if } d^{-1} \notin D \\
\dfrac{p(a)\cdot d^{-1}}{e_A} & \mbox{if } d^{-1}\in D \end{cases} \]
Note that in both cases we have $k^{-1}k \sim e_K$. Furthermore note that this can be done for any $k\in K-0$. And thus every element of $K$ has an inverse.
We will finally turn to addition on $K$ and define it as follows:
\[ \frac{d_1}{a_1} + \frac{d_2}{a_2} = \frac{p(a_2)d_1 + p(a_1)d_2}{a_1a_2} \]
Which is closed an commutative by closure and commutativity of multiplication on $A$ and $D$ and of addition on $D$. Therefore $K$ is a field.
Now let $i: D\to K$ be the homomorphism given by $i(d)=\frac{d}{e_A}$. Note that this homomorphism is injective. It is a homomorphism of rings with unit as $i(e_D) = \frac{e_D}{e_A} = e_K$. Furthermore there does not exist any subfield of $K$ that contains $i(D)$ by construction of $K$ (since any such subset would be missing some inverse of an element in $d$   - i.e. an element in $A$, or the product of such an $a$ with some element in $d$ and would thus not be a field). Thus $(K,i)$ must be a field of fractions for $D$ and thus every integral domain has a field of fractions. \par 
\subsection*{Uniqueness}
\begin{lemma}
If $(K,i)$ is a field of fractions for some integral domain $D$. Then we may find a surjective homomorphism $p: D\times D \to K$ such that $p(d_1,d_2)=k \iff k=i(d_1)(i(d_2))^{-1}$. 
\begin{proof}
$P$ is a homomorphism since $i$ is. We will now prove that it is indeed surjective.
Assume that for some $k$ there does not exist a tuple $(d_1,d_2)$ such that $k=i(d_1)(i(d_2))^{-1}$. Let $A$ be the set of all such $k$. Now consider $K-A$. Clearly $i(D)\subset K-A$, since we may just choose $d_2=e_d\so i(d_2)^{-1}=e_K$. Furthermore $K-A$ must be closed under multiplication (since $i$ is an isomorphism) and inversion (by construction of $K-A$), so $K-A$ is a proper subfield of $K$ if $A\neq \emptyset$. However since $K$ is a field of fractions, we must have $A=\emptyset$.
\end{proof}
\end{lemma}
Let $(K_1,i_1)$, $(K_2,i_2)$ be two fields of fractions for an integral domain $D$. Given $p$ as in the above lemma, we may construct an isomporphism by considering the map from the map of equivalence classes in $D\times D$ under $p$ to $K$. Now, by the construction of $p$,
\[ (d_1,d_2)\sim (d_3,d_4) \iff i(d_1)(i(d_2))^{-1}=i(d_3)(i(d_4))^{-1} \iff  i(d_1)i(d_4) = i(d_2)i(d_3) \]\[ \iff i(d_1d_4) = i(d_2d_3) \iff d_1d4=d_2d_3 \]. Note however that this is the same equivalence realtion as that used in the construction of $K$ above so for both $(K_1,i_1)$, $(K_2,i_2)$ there exists an isomorphism $m_1: K\to K_1$, $m_2: K \to K_2$ and thus we may construct an isomorphism $j: K_1\to K_2$ as $j=m_2\circ m_1^{-1}$. Note also that we may identify $m_1|_{i_k(D)}$ with $i_1$ (and the same holds for $m_2$) and thus we must have $j\circ i_1 = j|_{i_1(D)} \circ i_1 = i_2 \circ m_1^{-1}|_{i_k(D)} \circ m_1|_{i_k(D)} = i_2$ as desired.
\section*{Problem 1b)}
Note that for $\Z$ in the construction of $K$ in 1a), $A=D$ and thus $K=\Z\times \Z/\sim$. Furthermore, the equivalence relation is the same as that for $Q$ and thus we must have $K=\Q$ which is a field of fractions as constructed in 1a).
\section*{Problem 1c)}
The field of fractions over $K[X]$ is the collection of all rational functions (i.e. fuctions whose value is in the field of rational function over $K$. 
\section*{Problem 2}
Recall the definition of the the direct sum R-module $\bigoplus_{\alpha\in A} M_\alpha = \{ (m_\alpha) \in \prod_{\alpha\in A} M_\alpha | \text{ cofinitely many } m_\alpha = 0 \}$. To make this an $R$ module note that elementwise addition and elementwise multiplication of two sequences in the direct sum will again give an element in the direct sum.  Now let $T\in \Hom(\bigoplus_{\alpha\in A} M_\alpha, N)$. Then since $T$ is an homomorphism, we must have $T(m_\alpha+m_\beta)=T(m_\alpha) + T(m_\beta)$. Now note however, that we may write $m_\alpha = \sum_{\beta \in B} m_\beta$ where $B$ is a finite subset of $\alpha$ and $m_\beta \in M_\beta$ (which thus is an element in the direct sum where all but one entry are $0$).  Thus for every $m_\alpha$, $T(m_\alpha) = \sum_{\beta \in B_{m_\alpha}} T_\beta m_\beta$ where $T_\beta: M_\beta \to N$ is a homomorphism. Furthermore that if $m_{\beta,1}, m_{\beta,2} \in M_\beta$, we must have $T_{\beta,1}=T_{\beta,2}=T_{\beta}$ and we may find such a $T_\alpha$ for every $\alpha \in A$ and we may thus identify uniquely $T$ with the entry of the direct product whose $\alpha$-entry is $T_\alpha$. 
\section*{Problem 3a)}
\begin{obs}
Any combination of elements in $\Q(S)$ through multiplication, addition or inversion (which in in case is equivalent to reciprocation yields another element in $\Q(S)$ since $\Q(S)$ is a field and closed under these operations.. 
\end{obs}
We know that a line through two points $(x_0,y_0)$, $(x_1),(y_1)$ in $\R^2$ must satisfy the equation $ay+bx+c=0$ where 
\[ a=x_1-x_0, b=y_1-y_0m, c=ay_0-bx_0 \]
$x_0,x_1,y_0,y_1\in \Q(S)\so a,b \in \Q(S)\so c\in \Q(S)$
\section*{Problem 3b)}
We have two lines given by the two equations $0=a_1y_1+b_1x_1+c_1$ and $0=a_2 y_2+b_2 x_2+c_2$.  Since the lines are non-parallel, at least one of the lines has finite slope (without loss of generality assume this is the line given by the first equation), so we may write $y_1 = -\frac{b_1}{a_1} x_1 - \frac{c_1}{a_1}$. Since we require $y_1=y_2$ we must have 
 $-\frac{a_2 b_1}{a_1} x_1 - \frac{a_2 b_1}{a_1}  + b_2 x_2 + c_2 = 0$ and since we require $x=x_1=x_2$, we have
 \[ x = \dfrac{c_2 - \frac{a_2 b_1}{a_1}}{b_2 - \frac{a_2 b_1}{a_1}} \]
 Note that the denominator must be non-zero since the slopes of the lines are not equal, and since $a_i,b_i,c_i|_{i\in\{1,2\}} \in \Q(S)$ we must have $x\in \Q(S)$. 
\section*{Problem 3c)}
%Recall that if a circle has two points of intersection, the line of intersection is always orthogonal to the line through the origin of the circles (where orthogonality of two lines with equations $0=a_1y_1+b_1x_1+c_1$ and $0=a_2 y_2+b_2 x_2+c_2$ means that $a_1a_2 + b_1b_2 = 0$). Now, let $C_1,C_2 \subset \R^2$ be two circles centered at $s_1$ and $s_2$ and passing through $s_3$ and $s_4$ respectively. Then for any combination of these points we may find lines with coefficients in $\Q(S)$ connecting them. Let us write $r^2=a_{1,3}^2+b_{1,3}^2$, $R^2=a_{2,4}^2+b_{2,4}^2$ For the lines connecting $s_1$ to $s_3$ and $s_2$ to $s_4$, respectively. Now, let $\{(m,n),(l,k)\}  = C_1 \cap C_2$. A line through both of these points can be described by $a_{c} y + b_c x + c_c = 0$.
We have two circles $C_1$ and $C_2$ centered at $(x_1,y_1)$ and $(x_2,y_2)$ respectively passing through the points $(x_3,y_3)$ and $(x_4,y_4)$ respectively.
Then any point $(x,y)\in C_1\cap C_2$ will have to satisfy
\[(x-x_1)^2+(y-y_1)^2=(x_3-x_1)^2+(y_3-y_1)^2 \]
\[(x-x_2)^2+(y-y_2)^2=(x_4-x_1)^2+(y_4-y_1)^2 \]
Subtracting the second equation from the first, and canceling terms gives:
\[ 0 = 2(x_2-x_1)x+2(y_2-y_1)y + x_1^2 -x_2^2 + y_1^2-y_2^2 + x_3^2-x_4^2 + y_3^2-y_4^2 + 2(x_3-x_4)x_1+2(y_3-y_4)y_1 \]
Which (though messy) is a linear equation in and since all the $x_i,y_i$ correspond to points in $S$, all the coefficients are in $\Q(S)$ (note if $2\notin \Q(S)$ consider $2a=a+a$).
\section*{Problem 3d)}
Let $p$ be the point of intersection between of a circle centered at $s_1$ and passing through $s_2$ and a line given by the equation $ax+by+c=0$ such that the circle and the line intersect at some point $p=(p_x,p_y)$. The intersection point of the circle with the line are given by a quadratic equation with coefficients with in $\Q(S)$, so we know that $p_x^2\in \Q(S)$. If there is only one intersection point, it must be in $\Q(S)$. If there is two, if either one of them is, so is the other (due to the symmetry of the quadratic solution), so we only need to consider one (namely $p$).\par 
Now consider the following two cases. If either $p_x$ or $p_y$ is in $\Q(s)$, the n we may write down an explicit equation for the other simply by plugging it into the line equations and thus the other coordinate must be in $\Q(S)$ as well. If neither $p_x$ nor $p_y$ is in $\Q(S)$, let us extend $\Q(S)$ by one of them and then the other must be in the new field as well by the above argument. Let us choose $p_x$ for that extension. The set $\Q(S \cup \{p\})$ is $\Q(S)$ union any combination (under addition and multiplication) of $p$ with any of the of the elements in $\Q(S)$ or $\Q(S\cup \{p\}) = \Q(S) \times \Q(S)$ i.e. for any element $x\in \Q(S\cup \{p\})$, we may write $x=a+bp_x$. Then $x_1x_2=(a_1a_2+b_1b_2p_x^2)+2b_1b_2p_x$ and $x_1^{-1} =  \frac{a_1}{a_1^2+p_x^2a_2^2}-\frac{a_2}{a_1^2-p_x^2a_2^2}p_x$, both of which must be in $\Q(S\cup \{p\}$ since the coefficients are in $\Q(S)$ since $p_x^2$ is in $\Q(S)$. And thus $\Q(S\cup \{p\})$ is a field. Furthermore, note that $(a,0)$, $(0,bp_x)$ is a linearly independent spanning set and thus $\Q(S\cup \{p\})$ must have degree two over $\Q(S)$.
\end{document}