\documentclass[12pt,letterpaper]{article}

\usepackage{amsmath} 
\usepackage{amssymb}
\usepackage{ulem}
\usepackage{wasysym}
\usepackage{tikz}
\usepackage{multicol}
\usepackage{verbatim}
\usepackage[left=1in,top=1in,right=1in,bottom=1in,nohead]{geometry}
\usetikzlibrary{decorations.markings}
\usetikzlibrary{decorations.pathreplacing}

\usepackage{amsthm} 
\usepackage{wrapfig}
\usepackage{enumitem}
%\usepackage{enumerate}
\newtheorem{mydef}{Definition}
\newtheorem{example}{Example}
\newtheorem{thrm}{Theorem}
\newtheorem{lemma}{Lemma}
\newtheorem{cor}{Corollary}
\newtheorem{notation}{Notation}
\newtheorem{rem}{Remarks}
\newtheorem{obs}{Observation}
\newtheorem{claim}{Claim}
\newtheorem{term}{Terminology}
\newcommand{\biu}[1]{\underline{\textbf{\textit{#1}}}}
\newcommand{\so}{\Rightarrow}
\newcommand{\onlyif}{\Leftarrow}
\newcommand{\myiff}{\Leftrightarrow}
\usepackage[ampersand]{easylist}

\let\oldemptyset\emptyset
\let\emptyset\varnothing

\author{Keno Fischer}

\newcommand{\homework}{\biu{Homework}}
\newcommand{\Mor}{\text{Mor}}
\newcommand{\Hom}{\text{Hom}}
\newcommand{\clos}{\text{clos }}
%\newcommand{\Im}{\text{Im }}
\newcommand{\Ker}{\text{ker }}
\newcommand{\N}{\mathbb{N}}
\newcommand{\Q}{\mathbb{Q}}
\newcommand{\Z}{\mathbb{Z}}
\newcommand{\R}{\mathbb{R}}
\newcommand{\C}{\mathbb{C}}
\newcommand{\limn}{\lim\limits_{n\to\infty} }
\newcommand{\limk}{\lim\limits_{k\to\infty} }
\newcommand{\pabs}[1]{\left|\left| #1 \right|\right|_p}
\newcommand{\mesh}{\text{mesh}}
\newcommand{\set}[1]{\left{#1 \right}}
\newcommand{\paren}[1]{\left(#1 \right)}
\newcommand{\tensorp}{\bigotimes}
\newcommand{\tensor}{\otimes}
\newcommand{\Wedge}{\bigwedge}
\newcommand{\End}{\text{End}}
%\newcommand{\pabs}[1]{#1}
\begin{document}
\tikzstyle{lattice}=[shape=circle,draw,fill,text=white]
\tikzset{
  % style to apply some styles to each segment of a path
  on each segment/.style={
    decorate,
    decoration={
      show path construction,
      moveto code={},
      lineto code={
        \path [#1]
        (\tikzinputsegmentfirst) -- (\tikzinputsegmentlast);
      },
      curveto code={
        \path [#1] (\tikzinputsegmentfirst)
        .. controls
        (\tikzinputsegmentsupporta) and (\tikzinputsegmentsupportb)
        ..
        (\tikzinputsegmentlast);
      },
      closepath code={
        \path [#1]
        (\tikzinputsegmentfirst) -- (\tikzinputsegmentlast);
      },
    },
  },
  % style to add an arrow in the middle of a path
  end arrow/.style={postaction={decorate,decoration={
        markings,
        mark=at position 1 with {\arrow[#1]{stealth}}
      }}},
}

\begin{term}
A complex polynomial is a formal expression of $\sum\limits_{k=0}^N a_kx^k, a_0, \ldots, a_N \in \C$.
\end{term}
Suppose $P(x)\in \C[x] =_{def}$ ring of complex polynomials. Then for any $z\in \C$, we can make sense of $p(z)\in \C$. Now $z\to p(z)$ is a polynomial function.
\begin{obs}
$z\to p(z)$ is a continuous function from $\C\simeq \R^2$ to $\C$. Also $z\to |z|$ is a continous map from $\C$ to $\R$. Then $|p(z)|$ is continuous.
\end{obs}
\begin{term}
One calls $p(x)-\sum_{k=1}^n a_k x^k$ constant if $a_k=0$ for $k>0$. \par Furthermore one calls $p(x)$ monic if $a_N=1$
\end{term}
\begin{thrm}[Fundamental Theorem of algebra]
Every non-constant polynomial $p(x)$ has a complex root.
\begin{proof}
In proving the FTA, we may and shall suppose without loss of generality
\begin{enumerate}
\item $p(x)$ is monic
\item The degree is at least $2$. 
\end{enumerate}
\begin{lemma}
For every $C>0$, there exists $R>0$ such that $|z|\geq R \so |p(z)| \geq C$
\begin{proof}
We may suppose that $R\geq 1$. Then for $|z|\geq R$, $|p(z)|\geq |z|^N - \left| \sum\limits_{k=0}^{N-1} a_k z^k \right|$ by the triangle inequality. However this must be $\geq |z|^N - \sum\limits{k=0}^{N-1} |a_k||z|^k \geq |z|^N - A|z|^{N-1}$, where $A=\sum\limits_{k=0}^{N-1} |a_k|$. Now make $R\geq \max(1,2A)$. Thus we have 
\[ = (|z|-A)|z|^{N-1} \geq \frac{1}{2} |z|^N \geq \frac{1}{2}R^N\]. So we need $R\geq \max(1,2A,(2C)^{\frac{1}{N}}$.
\end{proof}
\end{lemma}
\begin{cor}
There exists $z_0\in \C$ such that $|p(z)|\geq |p(z_0)|$ for all $z\in \C$. 
\begin{proof}To see this, choose $C=|p(0)|$. Then with $R$ as in the Lemma, $\{z\in \C | |z|<R\}$ is compact. Thus there must exist $z_0$ such that $|z_0|\leq R \so |p(z)|\geq |p(z_0)|$. Since   the set is compact, we may find an $z_0$ such that $|p(z_0)|$ forms a lower bound of $|p(z)|$ where z in the above compact set.\par 
On the other hand, $|z|>R\so |p(z)|\geq|p(0)|\geq |p(z_0)|$ since $0$ is in the above set.
\end{proof}
\end{cor}
Now, if the FTA were to fail, there would exist $p$ as above s.t. $p(z_0) \neq 0$ - if not replace $p$ by by $q(x) = p(x+z_0) = \sum_{k=0}^N a_k(x+z_0)^k$. Note that the sum can be expanded by the binomial theorem. \par 
Then $p(z_0) = a_0 \neq 0$. Replace $p(x)$ by $\frac{1}{a_0}p(x)$. Then $p$ is no longer monic, but otherwise has the same properties plus $p(0) = a_0 = 1$. Thus we have $p(z) = 1+\sum_{k=1}^N a_k z^k$. Choose $m$, $1\leq m \leq N$ such that $a_1=\ldots=a_{m-1}=0$, $a_m\neq 0$. \par
Then $p(z) = 1 + a_mz^m + \sum_{k=m+1}^N a_k z^k$. To arrive at the desired contradiction, we shall now construct some $z$ such that $p(z)<p(0)=1$.\par 
We shall only consider values of $z=re^{i\theta}$, $0<r<1$, so that $a_mz^m \in R_{<0}$. We can do this by fixing $\theta$ such that $m\theta + \text{arg} a_m = \pi$. Then, for such $z$,
\[ |p(z)| = | 1-|a_m|r^m + \sum\limits_{k=m+1}^N |a_k| r^k| \leq 1-|a_m|r^m + \sum\limits_{k=m+1}^N |a_k| r^k \leq 1-|a_m|r^m + Ar^{m+1} = 1 - (a_m - Ar)r^m \] 
Provided $r\leq (a)$l
\end{proof}
\end{thrm}
\end{document}